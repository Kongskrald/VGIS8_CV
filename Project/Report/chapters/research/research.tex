\chapter{Research}
\label{cha:Research}

\section{Iris Recognition}
Modern iris recognition started in with an article by John Daugman \cite{Daugman1993} discussing the security of using iris for recognition. An outline of how to do recognition was also laid out with the introcution of using  2D Gabor Waveletts to extract features of the complex pattern that can be found in an iris. This methodology has since been the basis for most iris recognition. It was also the foundation for IrisCode which is a commercially developted iris recognition algorithm by John Daugman. In 2016 a handbook for iris recognition \cite{Bowyer2016b} was published giving an outline of the whole process of iris recognition. In general Near Infra-Red (NIR) images of an iris are used but other type of images can also be used.  \cite{Khan2017a} show how to use Daugmans methodology on iris images taken with a smartphone in visible light. They use Daugmans Integro-differential operator to localize the bounds of the iris. Then they suppress the eyelids the eyelids which often cover parts of the iris by using an apporach inspired by Masek. Afterwards the image is normalized by using the humogenous rubber sheet model by Daugman. Then eyelashes are removed from the image and feature extraction is done by using 2D Gabor Waveletts. This approach to extract features in one way or another can be seen in multiple state of the art iris recognition systems and research papers e.g \cite{Luhadiya2017a} \cite{Uka2017a} \cite{Kuehlkamp2016a}.\cite{RifaeeMustafaandAbdallahMohammadandOkosh2017a}