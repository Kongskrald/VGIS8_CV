\section{Information Fusion}

A system that strives to utilise information from two or more different biometric traits in order to obtain one result is a called a multi-modal which is a kind multi-biometric system. Besides the multi-modal approach there are several other approaches, which results in information from two or more sources, therefore, methods for fusion of information from different sources into one system for classification purposes is a widely investigated area. \citep{Bowyer2016b} \todo{should this be in a small introduction before the three different areas?}

Fusion of information can happen on different levels. It can happen on one of five levels, each with a higher level of preprocessing: signal level, feature level, score level, rank level, or on the decision level. The level on which the fusion should happen depends on the kinds of multi biometric data dat has to be fused, and purpose of the fusing. In general score-level and feature-level are the most popular fusing techniques \citep{Bowyer2016b}. Fusing on the lowest level, signal level, might be done in order to merge data from different sources to construct a more detailed or larger dataset or signal. On the feature level the fusing might happen through merging of extracted features from different sources into one feature vector \citep{Ross2003}. On the score level it can be determining the best sample to use for the processing based on which has the highest score and is the best match to the gallery samples. Rank level can be similar to the scores but dependent on match rankings, on decision level it can be making the decision based multiple classifiers, e.g. one for each modality\citep{Fierrez2018b}.

In literature a large variety of methods and algorithms have been utilised for fusing information on different levels. Some algorithms are fairly simple and basically makes decisions on which sample to use onwards for classification based on matching scores between samples and the gallery. Other methods are more advanced and well known for use in a variety of applications. These are methods such as SVM, k-NN, decision trees, and bayesian methods \citep{Ross2003}. The highest rank method and Borda count are both well known methods for combining information based on ranking. The highest rank method ranks possible classes based on the highest rank assigned to the class by a classifier across all classifiers.\citep{Ho1994} The Borda count is a kind of majority voting which can be used in combination Multiple Classifier Systems\citep{Bowyer2016b,Ho1994}.  

\section{Multi-Modal Databases}
Even though recognition based on biometric traits is widely investigated, and research shows that multimodal systems perform better than the uni-modal systems based on the same data, the research in this area is limited and incomplete \citep{Chen2005a,Bowyer2016b}. Because of the limited availability of multimodal datasets, such datasets are often synthetically constructed based on randomly combined data, eg. iris and face datasets \citep{Chen2005a}. Only a limited amount of studies on actual utilise a multimodal dataset obtained from the same test subjects or maybe even with one sensor. However, a few named multimodal datasets are encountered in literature. Examples hereof are the dataset $IV^2$, consisting of data obtained from 300 subjects, and the datasets provided for the The Multiple Biometric Grand Challenge, consisting of \citep{Petrovska-Delacretaz2008a,Bowyer2016b}. The latter is available in two versions and can be obtained on request. Furthermore, it serves as a common test set in order to compare performance. However, some multimodal datasets have been created by researchers during their work, which is not generally available. 

