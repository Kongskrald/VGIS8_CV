

\section{Information fusion and face-iris databases}

A system that strives to utilise information from two or more different biometric traits in order to obtain one result is a called a multi-modal which is a kind multi-biometric system. Besides the multi-modal approach there are several other approaches, which results in information from two or more sources, therefore, methods for fusion of information from different sources into one system for recognition purposes a widely investigated area.  The fusion can happen on different levels. It can happen on one of five levels: signal level, feature level, score level, rank level, or on the decision level. In general score-level and feature-level are the most popular fusing techniques \citep{Bowyer2016b}. For each of the fusing techniques there are several approaches to how to do the fusing. For the lowest level, signal level, the fusing might be merging data from different sources to construct a more detailed dataset or signal. On the feature level the fusing might happen through merging of extracted features from different sources into one feature vector \citep{Ross2003}. On the score level it can be determining the best sample to use for the processing based on which is has the highest score and is the best match to the gallery samples. Rank level can be similar to the scores but dependent on match rankings, on decision level it can be applying Multiple Classifier Systems, e.g. one for each modality\citep{Fierrez2018b}. 
Even though recognition based on biometric traits is widely investigated, and research shows that multimodal systems perform better than the uni-modal systems based on the same data, the research in this area is limited and incomplete. Because of the limited access to multimodal datasets, such datasets are often synthetically constructed based on randomly combined iris and face datasets. Only a limited amount of studies on actual utilise a multimodal dataset obtained like that. The named multimodal datasets encountered in literature are the dataset $IV^2$ and the datasets provided for the The Multiple Biometric Grand Challenge\citep{Bowyer2016b,Petrovska-Delacretaz2008a}. The latter is available on request and also serves as a common test set in order to compare performance.  