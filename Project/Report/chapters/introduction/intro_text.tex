The problem of protection of physical elements or information by limiting the access to only the relevant people is well known. The central challenge to the problem of security is the challenge of verifying the identity of a person trying to acquire access. The emergence of biometric techniques has induced an increasing interest in biometric-based security rather than knowledge-based or token-based security. This is mainly due to the fact  that the more traditional methods for security systems are easier breached or spoofed \citep{Ross2003}. Through the last decades researchers have investigated identity verification based on different biometric modalities. In the last decade investigations have been conducted in combining several biometric modalities in one system with the purpose of creating a system that performs better than the ones only utilising a single one. Results show that combining modalities performs better than any of the modalities separately \citep{Chen2005a}.  However, increased accuracy is not the only benefit of utilising multiple biometric traits. More modalities increase the universality of the system and decrease the influence of noisy measurements \citep{Ross2003}.

One of the fields where biometric-based security is increasingly applied is security for handheld devices such as smart phones. Although technology is advancing and mobile devices are equipped with still more advanced components, the computing power and the quality of the data from sensors, are both constraints of mobile devices. These limiting factors make it more challenging to make successful biometric-based identity verification on mobile devices rather than in other applications \citep{Kim2016}. Though the use of machine learning for image processing is well known, it has only scarcely been applied on data obtained by cameras on mobile devices. \cite{Khan2017a} presents different machine learning methods applied on iris images obtained by a smart phone. \cite{Bazrafkan2017} applies deep learning for segmentation purposes on a database containing images acquired using a smart phone among others. 

Furthermore, the application context of mobile devices, or more specifically smart phones, introduces some limitations in regard to what biometric traits can be used. Due to the compact design and the sensors commonly incorporated in the devices, acquisition of only a few different traits is possible. The fact that the user of the device should be able to easily and conveniently acquire the data of them selves delimits the possibilities even further. The biometric traits, which adhere to the limitations are traits such as face, iris, voice, fingerprint, and signature. Due to the arguments found in literature that iris is very distinctive, while face is non-invasive, these are chosen as the biometric traits used for the solution \citep{Wang2009a}. 

Based on these arguments, this work strives to make a system for identity verification based on the biometric traits of iris and face for use on mobile devices.