\chapter{Future Work}
\label{ch:future_work}
This chapter casts light on some of the considerations and potential improvements, which could be considered during prospective work. 

Although fusion of the two modalities was accomplished there are still possibilities to investigate. 
In \autoref{ch:intro} different benefits of multimodal biometric systems are mentioned, some of them being increased safety, universality and reduced noise sensitivity. The latter two being important points in regard to usability of the system. In some cases people might not be able to provide certain data, or the data provided might be noisy because of imperfect data capture. Systems utilising few or only one modality are especially sensitive in those cases. 

In the current implementation of the fused \gls{cnn} the left and right irises of the same subject are considered separately as two different classes. This was done because of the fact that the two irises are structurally different even though they might have the same colour. In future work, a fusion net could be designed in such a way that it fuses three inputs and thus fuses face data with data from both irises of the same person. This might be a way to obtain more information from one subject with the means that are already used. This could make the system more secure, but also increase the chances of classification when the inputs are noisy. 

Another point that might be worth considering for future work is the database used. The dataset used for the fusion was, as mentioned, a synthetic database comprising the two databases \gls{lfw} and Warsaw-Biobase. However, in literature it is suggested that this is not representative of how a real multimodal biometric database obtained from the same subject because of the natural correlation between the biometric traits recorded. Therefore the performance implemented system might be most realistically represented when trained and tested on a genuine multimodal biometric dataset. 

Furthermore, as \gls{cnn}s are best trained on large datasets it might be interesting to train on a larger dataset in order to investigate whether this improves the results. 


