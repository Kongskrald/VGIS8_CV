\chapter{Conclusion}\label{ch:conclusion}\glsresetall
Throughout the project, the work has been aimed at finding a solution to the problem statement; \\
\textit{How well can identity verification be performed on \gls{vl} smart phone images of iris and face, and to which extend can the performance be improved by information fusion?}

To be able to perform identity verification on smart phone images of iris and face, several recognition solutions were made.

Iris recognition was done in multiple ways. Firstly, different regular machine learning solutions were made. The best performance was achieved using a polynomial kernel for \gls{svm}, which classified with an accuracy of $98\%$. However, this did not meet the requirement set for the accuracy. Therefore, a deep learning solution was subsequently introduced. By using a \gls{cnn}, with a total of 14 layers, it was possible to achieve an accuracy of $99.7\%$, which satisfies the requirement of at least $99\%$ accuracy. 
Both the machine learning and deep learning solution used the Warsaw-BioBase database, a database with close-up photos of irises captured with an Apple iPhone 5s.

Face recognition was made using a pre-made image classification model, VGG16. This model is pre-trained on the ImageNet database with 2000 different classes, and the weights from this are used in the implementation of the model. The \gls{cnn} used the \gls{lfw} database and achieved an accuracy of $99.35\%$, which also satisfies the requirement of accuracy limit.

A fusion of the two successful \gls{cnn}s was then made in order to examine, whether, information fusion could improve the performance. For the fusion network, the databases used for the two separate networks was synthetically combined into one chimeric database with a label for samples comprising an iris and a face tying them together. This fusion network achieved an accuracy of $81.17\%$, which is not better than any of the stand-alone networks. When the iris and face stand-alone networks were trained using the chimera data the accuracies of these dropped to $68.94\%$ and $77.79\%$ respectively. This could indicate that while the chimera data behaves unexpectedly, the fusion net still performs better than the stand-alone networks when they are trained on the same data.