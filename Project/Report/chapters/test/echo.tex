\section{Echo}
The echo is tested both regarding to \autoref{req:DL1} and \autoref{req:DL3} and the effect description in \autoref{sec:effect_descr}. 

The complete test is shown in \autoref{ch:echo_meas_app}. The plots of the results are shown in \autoref{fig:echo_plot_4410} and \autoref{fig:echo_plot_dec500}. $4410$ samples is \SI{100}{\milli\second}, $2500$ samples is \SI{56,7}{\milli\second} and $29\,999$ samples is \SI{680}{\milli\second}.

\begin{figure}[hbpt]
	\centering
	\includegraphics[width=\textwidth]{echo_plot_del4410}
	\caption{The two plots show the impulse echoed with a delay of 4410 samples and decays of 500 and 900.}
	\label{fig:echo_plot_4410}
\end{figure}

\begin{figure}[hbpt]
	\centering
	\includegraphics[width=\textwidth]{echo_plot_dec500}
	\caption{The two plots show the impulse echoed with a delays of 2500 and 29\,000 samples and a decay of 500.}
	\label{fig:echo_plot_dec500}
\end{figure}

The plots show that the echo effect does as described in \autoref{sec:effect_descr} by repeating the input signal with a lower amplitude making it fade out. This can be varied by adjusting both the decay and the delay of the signal.
This means both the requirements are fulfilled as long as the delay is more than $ 2205 $ samples, which is \SI{50}{\milli\second}. Thus the echo effect is concluded successful.