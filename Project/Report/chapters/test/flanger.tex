%!TEX root = ../../master.tex


\section{Flanger}\label{sec:flanger_test}


The test of the flanger is made according to \autoref{req:DL2} and the effect description in \autoref{sec:flanger_design}.
It is shown in \autoref{fig:test_flanger_s} that the variable comb filter filters out the data at the frequencies where the sinusoids overlap. This changes over time as the delay changes with the \gls{lfo}.
\begin{figure}[hbpt]
	\centering
	\includegraphics[scale=0.8]{test_flanger}
	\caption{Spectrogram of white noise sent through flanger.}
	\label{fig:test_flanger_s}
\end{figure}

The conditions for both the figures in \autoref{sec:flanger_design} and \autoref{fig:test_flanger_s} are the same which means that the delay time is 1 \si{\milli\second}, the depth is 0.7 \si{\milli\second} and the pre-delay is 1 \si{\milli\second}. Comparing the simulated graph and the graph from the test yields the same results as seen when comparing \autoref{fig:freq_flanger} and \autoref{fig:test_flanger_s}. Thus the test is evaluated as successfull 

