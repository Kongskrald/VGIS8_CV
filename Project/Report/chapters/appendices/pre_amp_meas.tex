\chapter{Pre-amplifier measurements}\label{pre_amp_meas_app} 
\section{Purpose}
It is wanted to measure the \gls{snr} in order to test if \autoref{req:qr1} is fulfilled. The gain in the \gls{preamp} is also measured in order to test if \autoref{req:input_signal} and \autoref{req:qr3} are fulfilled.

\section{List of equipment}

\section{Set-up}

\section{Procedure}
\paragraph{Signal to noise ratio}
The output from the \gls{preamp} is measured using an oscilloscope with an input signal of know amplitude and frequency. The data from the oscilloscope is saved to csv file and the frequency spectrum is plotted in MATLAB using the \gls{fft} command. The spectrum is then analysed according to frequencies corresponding to the original signal.
\paragraph{Output signal}
A signal of known amplitude is put on the input of the \gls{preamp}, and the output is measured. The gain is then varied to measure the maximum and minimum gain in the \gls{preamp}.



\section{Results}