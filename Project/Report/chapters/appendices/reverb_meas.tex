\chapter{Reverb Audio Effect Measurements}\label{reverb_meas_app} 
Reverb is tested in regard to both the requirements in \autoref{sec:reqDL} and the analysis in \autoref{sec:effect_descr}.


\section{Purpose}
It is necessary to measure if the reverb effect designed and implemented meets the requirements set. According to \autoref{req:DL1} the reverb's delay time must be greater than the time interval of \SI{50}{\milli\second} given in \autoref{tab:delay_times}.  

\section{List of Equipment}
\begin{itemize}
	\item Digilent Analog Discovery 2 including BNC-adaptor board
	\item Philips PM 5715 pulse generator (AAUnr. 08644)
	\item TMS320C5515 eZdsp
	\item A computer
\end{itemize}

\section{Set-up}
The pulse generator sends a single impulse into the \gls{dsp}. The output signal from the \gls{dsp} is read on the Digital Analog Discovery 2 and analysed. By using the Waveforms software set at single input, only one impulse is recorded and saved at a time.

\section{Procedure}
A single impulse is sent in on the input of the \gls{dsp}. The output from the \gls{dsp} is then plotted and analysed according to both delay and amplitude.

The impulse is sent with the Philips PM 5715 pulse generator. This is set to do single shot, which is controlled by the user. The duration needed is \SI{1}{\milli\second} to intercept the signal on the oscilloscope. The ramp is set at the lowest possible value of \SI{6}{\nano\second}.

\section{Results}
The tap delay lines are set at $ 2205, 5851, 9143, 12466, 17081 $ and $ 23373 $ samples. The result of this is shown in \autoref{fig:reverb_plot_app}.
$2205$ samples is exactly \SI{50}{\milli\second} which is also seen in the figure from the first top to the second. This means that \autoref{req:DL1} is fulfilled, as this is the lowest delay of the tap delay lines.

\begin{figure}[htbp]
	\centering
	\includegraphics[width=\textwidth]{reverb_plot}
	\caption{}
	\label{fig:reverb_plot_app}
\end{figure}

The figure shows the six early reflections as the six first delays and the following delays which are the late reflections. Just as simulated in \autoref{sec:design_reverb}. This means the effect is acting as intended and therefore successful.