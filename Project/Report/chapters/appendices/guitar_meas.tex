%!TEX root = ../../master.tex
\chapter{Guitar Measurements}
\label{ch:guitar_meas}

\section{Purpose}
As configurations of an output from an electric guitar is made, it is necessary to analyze the guitar and the properties of it.

This is done for several different settings. Since there are multiple pickups on the guitar the measurements are done for each setting of the pickup selector. 

\section{List of Equipment}
\begin{itemize}
	\item Fender Stratocaster 
	\item NI-4461 - sine sweeping computer
	\item $1 \text{ k}\Omega$ resistor ($R_L$)
	\item Oscilloscope (AAUnr.: 33854)
 \end{itemize}

\section{Set-up}
\paragraph{Output Impedance}
To measure the output impedance the test scenario is set up as seen in \autoref{fig:impedance_meas_block}. The switch shown in the figure is not an actual switch, but a cable that can be connected or disconnected depending on which measurements are desired. To connect the input and the output from the NI-4461 these are connected to the jack input on the guitar via a TS (tip sleeve) to a BNC converter where the input from the computer is connected to the tip and the output is connected to the sleeve. 

\begin{figure}[htbp]
	\centering
	\includegraphics[scale=0.6]{figures/appendix/measuring_figure.pdf}
	\caption{A block diagram showing the setup for measuring the guitars output impedance.}
	\label{fig:impedance_meas_block}
\end{figure}

\paragraph{Output Voltage}
To measure the output voltage the guitar is connected directly to an oscilloscope. The oscilloscope is set to measure the peak-to-peak value of the signal.

\section{Procedure}
\paragraph{Output Impedance}
The same procedure is done with and without the tone control turned down for every position on the pickup switch which make the pickup combination as follows where pickup 1 is closest to the bridge, pickup 2 is in the middle and pickup 3 is closest to the neck. 

\begin{itemize}
	\item position 1: pickup 1
	\item position 2: pickup 1 and 2
	\item position 3: pickup 2
	\item position 4: pickup 1 and 3
	\item position 5: pickup 3
\end{itemize}


Two measurements are made for each position, one with the load resistor (switch on) and one without (switch off). A sine sweep from 20 Hz to 20 kHz with an amplitude of 1 V is send from the computer through the guitar and back into the guitar which compares the input and output of the computer and gives the frequency response. The frequency response is then used to calculate the impedance at the different frequencies. This is done in \autoref{eq:da_very_label}.

\begin{align}
	&V = A\cdot cos(\frac{P}{180\cdot\pi})+j(A\cdot sin(\frac{P}{180\cdot\pi}) \\
	Where:& \nonumber \\
	&V = \text{output voltage in rectangular form} \nonumber\\
	&A = \text{Amplitude from frequency response} \nonumber\\
	&P = \text{Phase from frequency response} \nonumber 
	\label{eq:da_very_label}
\end{align}

The calculation from \autoref{eq:da_very_label} is made for V (without load) and $V_L$ (with load) which is used in \autoref{eq:impedance_calculation}.  

\begin{align}	
	&Z_O=\frac{V-V_L}{\frac{V_L}{R_L}} \label{eq:impedance_calculation} \\
	Where:& \nonumber \\
	&Z_O = \text{Output impedance} \nonumber
\end{align}	

\paragraph{Output Voltage}
The output voltage of the guitar is measured by strumming the guitar strings as hard as possible, giving a maximal output voltage. This is well above regular strumming, but will yield a \textit{worst case} value.

\section{Results}
\paragraph{Output Impedance}
The output impedance with and without the tone control are plotted at frequencies from 20 Hz to 20 kHz and is shown in \autoref{fig:impedance_meas_tone_result} and \autoref{fig:impedance_meas_result}. It is seen that position 1, 3 and 5 are close to equal and position 2 and 4 are close to equal. The reason for this is that at position 1, 3 and 5 the impedance is measured on one pickup and at position 2 and 4 it is measured at a combination of two pickups.  
\begin{figure}[htbp]
	\centering
	\includegraphics[width=0.7\textwidth]{figures/analysis/impedance_plot_tone.eps}
	\caption{A plot of impedance levels over frequency calculated for all pickup settings with both tone controls turned up.}
	\label{fig:impedance_meas_tone_result}
\end{figure}

\begin{figure}[htbp]
	\centering
	\includegraphics[width=0.7\textwidth]{figures/analysis/impedance_plot.eps}
	\caption{A plot of impedance levels over frequency calculated for all pickup settings with both tone controls turned down.}
	\label{fig:impedance_meas_result}
\end{figure}

The maximum output impedance from the guitar is read as $89 \text{ k}\Omega$ at 5 kHz. Since the input impedance of the \gls{dsp} development board, which is the input impendance of the \gls{adc} located at the input is \SI{20}{\kilo\ohm} \citep{dspadc}, a buffer between the two systems is required, in order to have an ideal ratio between the two systems. 

\paragraph{Output Voltage}
The highest value measured is \SI{1300}{\milli\volt} or \SI{1.3}{\volt}. This is measured in peak-to-peak values.
This will give an RMS value of \SI{0.460}{\volt}. 