%!TEX root = ../../master.tex
\chapter{Signal to Noise Ratio Measurements} 
\label{cha:signal_to_noise_ratio_measurements}


\section{Purpose}
To set \autoref{req:SNratio} the \gls{snr} is measured through the \gls{dsp} without any modification on the signal. This will be the reference for the \gls{preamp} and the effects when the \gls{snr} is measured. The purpose of knowing the \gls{snr} is to make sure there is not produced any noise, which will be audible and overrule the sound of the guitar when the effects are used.

\section{List of Equipment}
\begin{itemize}
	\item Digilent Analog Discovery 2 including BNC-adaptor board
	\item TMS320C5515 eZdsp
	\item A computer
\end{itemize}

\section{Set-up}
To run the test of the effects the \gls{dsp} is connected to the Analog Discovery 2 which is connected to a computer. The Analog Discovery 2 is used to send a sinus wave carrier signal into the \gls{dsp} where the effects are created and then send back into the Analog Discovery 2, as seen in \autoref{fig:signal_noise_effect}, which then creates a spectrum and compares the carrier signal to the noise and calculates the \gls{snr} from that.

\begin{figure}[hbpt]
	\centering
	\includegraphics[scale=0.8]{figures/appendix/effects_sn_test.pdf}
	\caption{The set-up for measuring the \gls{snr} of the effects.}
	\label{fig:signal_noise_effect}
\end{figure}

The same set-up is used to test the \gls{preamp} as shown in \autoref{fig:signal_noise_preamp}.

\begin{figure}[hbpt]
	\centering
	\includegraphics[scale=0.8]{figures/appendix/preamp_sn_test.pdf}
	\caption{The set-up for measuring the \gls{snr} of the \gls{preamp}.}
	\label{fig:signal_noise_preamp}
\end{figure}


\section{Procedure}
A 1 kHz sine wave signal is sent from the Analog Discovery 2 through the \gls{preamp} and back into itself. Using the software included with the Analog Discovery 2 a spectrum of the signal from the \gls{preamp} is made, and the \gls{snr} can be specified from the program. The program is set to minimum hold and a 1000 measurements are made to get a good representation of the \gls{snr}. 

\section{Results}
As result the \gls{snr} of the different effects and the \gls{preamp}.

\begin{itemize}
	\item Pass-through $\approx \SI{44}{\decibel}$	
	\item Echo $\approx \SI{45}{\deci\bel}$ 
	\item Reverb $\approx \SI{51}{\decibel}$
	\item Equaliser $\approx \SI{51}{\decibel}$
	\item \gls{preamp} $\approx \SI{58}{\decibel}$
\end{itemize}

The spectrum of the measurements is shown in \autoref{fig:signal_noise_spectrum_all} to show how the carrier signal compares to the noise.

\begin{figure}[htbp]
	\centering
	\includegraphics[scale=0.8]{figures/test/noise_spectrum.eps}
	\caption{Spectrum for all the measurements in one plot.}
	\label{fig:signal_noise_spectrum_all}
\end{figure}