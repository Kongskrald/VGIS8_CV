\section{Guitar Analysis}\label{sec:guitar_analysis}
The basics of any guitar is, that the strings vibrate at a certain frequency, and this generates a tone. For an acoustic  guitar the vibrations from the strings are amplified by the body of the guitar, which is hollow. This is the key difference between an acoustic guitar and an electric. Since an electric guitar does not have a hollow body which can amplify the tone of the strings, a pickup is used instead \citep{elecguit}. 

\subsection{Strings}
A standard guitar has six strings, and when you pluck the strings, they will vibrate at a certain frequency. If the guitar is standard tuned, the six strings' respective frequency are given as seen in \autoref{tab:freq_strings}.

\begin{table}[httb!]
	\centering
	\caption{Guitar strings' respective frequencies for a standard tuned 6-string guitar \citep{harmonics}.}
	\label{tab:freq_strings}
	\begin{threeparttable}
		\begin{tabularx}{\textwidth}{X c c X}
			\textit{String number} & \textit{String tone}  & \textit{Frequency} [\SI{}{\hertz}] & \textit{Frequency rounded}[\SI{}{\hertz}]\\ \toprule \rowcolor{lightGrey}
			6 & E & 82.407 & 82\\
			5 & A & 110.000 & 110 \\ 
			\rowcolor{lightGrey}
			4 & D & 146.832 & 147 \\
			3 & G & 195.998 & 196 \\
				\rowcolor{lightGrey}
			2 & B & 246.942 & 247 \\ 
			1 & E & 329.628 & 330\\
		\end{tabularx}
	\end{threeparttable}
\end{table}

This vibration of the strings, is the key element of generating a tone. The frequencies seen in \autoref{tab:freq_strings} are called the fundamentals or the first harmonic of the strings' pitch. This is the frequency generated, when you pluck an open string, and is shown in the first illustration in \autoref{fig:harmonics}. When the length of a string is halved, the frequency is doubled, and this gives a tone pitch which is similar to the full string length pitch, but it is an octave higher.

\begin{figure}[H]
	\centering
	\includegraphics[width=0.7\textwidth]{harmonics}
	\caption{Illustration of strings vibrating at the fundamental pitch, and the second and third harmonic made from \citep{pickup}.}
	\label{fig:harmonics}
\end{figure}

When looking at the vibration of the string, it will only be able to vibrate at whole number multiples of the fundamental pitch, which is called harmonics. It is hard to see when looking at a string, but there are fat and skinny spots on the string, when it is vibrating. The fat spots are called antinodes and the skinny spots nodes. An antinode and a node is the point of maximum and minimum movement of the string, respectively. This can be seen on the second illustration in \autoref{fig:harmonics}. The key part about a node is, that there is no movement at a node, and therefore you can touch the string at a note without changing the pitch. The key thing about an antinode is that the number of antinodes on a vibrating string is the number of the harmonics. So as shown on the third illustration in \autoref{fig:harmonics} this is the third harmonic of the string, since there is three antinodes \citep{pickup}.

When understanding the harmonics of a guitar, a better understanding of which frequency band a standard guitar works in is necessary. Consider string 1 in \autoref{tab:freq_strings}, an electric guitar usually has 22 bands which each lifts the tone one semitone, then if string 1 is plucked at band 22, a tone which is three octaves above the fundamental of string 4 is generated. This will produce a frequency of \SI{1176}{\hertz}. It is possible to produce even higher notes on a guitar, if it is plucked in a specific manner. This will produce a frequency of \SI{5873}{\hertz}. By doing special grips on the guitar, it is possible to produce frequencies throughout the audible frequency range, and even frequencies beyond what is audible to the human ear \citep{high_note}. 

\subsection{Pickup}
As explained, a pickup is used in an electric guitar, in order to generate a tone. A pickup is any transducer which captures or senses vibrations produced by a musical instrument, and therefore there are many types of pickups. The most commonly used type for electric guitars is magnetic pickups, and therefore this is the type of pickup reviewed in this section.

A very basic pickup consists of two main components, a magnet and a coil, where the magnet is the core of the pickup, and the coil is wrapped around the core. The basic science behind the function is Faraday's Law of Induction, which states that a changing magnetic field causes an electric field to be set up in a nearby wire, and thus causing a current to flow. This is if the wire is part of a closed circuit. The strings of an electric guitar are made up of materials that are ferromagnetic, which means they are attracted to magnets. When the strings then vibrate near the magnetic field of the pickup, they disturb the field of the pickup. This change in the magnetic field will make a current flow in the coil, which will track the vibration of the strings, and the pickup is working. This is shown in \autoref{fig:pickup_magnet}.

\begin{figure}[htbp]
	\centering
	\includegraphics[width=0.8\textwidth]{pickup11}
	\caption{Simplified illustration of a string in the magnetic field of the permanent magnet made from \citep{pickup}.}
	\label{fig:pickup_magnet}
\end{figure}

As explained, there are nodes and antinodes on a vibrating guitar string. Since there is no movement at a node, the pickup will not be able to generate a signal. Therefore it is wanted to have the pickups located under an antinode, since the signal generated will be more powerful. As the antinodes are not fixed, a number of pickups are usually used, so there is  at least one pickup located under or close to an antinode \citep{pickup}.

To further investigate the output signal produced by an electric guitar's pickup, measurements of the guitars output impedance and the maximum voltage output are measured. The measurements are shown in \autoref{ch:guitar_meas}. 
