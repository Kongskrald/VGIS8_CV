%!TEX root = ../../master.tex
\subsection{Flanger}
\label{ssub:flanger}

The flanger effect was introduced in the 1960's and was originally an analogue effect. The effect was made by using two tape machines playing two different tapes with the same music but in sync and the output was summed together. To create the effect a flange on one tape machine was touched lightly to create a small delay between the two tapes. The flange was then released and the same procedure was done on the other tape machine to counter the delay and create a delay on the other tape. The effect is often described as the sound of a jet plane passing by which makes the listener hear the direct sound from the plane but also a reflected sound from the ground beneath. 

Nowadays the flanger effect is mostly created digitally, as described in the block diagram in \autoref{fig:flanger_blockdiagram} and is modeled as a feed forward comb filter with a delay, $M$, which is varied over time. The output from the filter will then be as in \autoref{eq:flanger_out}.

\begin{figure}[htbp]
	\centering
	\includegraphics[width=0.8\textwidth]{figures/analysis/flanger_blockdiagram.pdf}
	\caption{Flanger block diagram.}
	\label{fig:flanger_blockdiagram}
\end{figure}

\begin{equation}
	y(n)=x(n)+g \cdot x(n-M(n))
	\label{eq:flanger_out}
\end{equation}

\startexplain
	\explain{$y(n)$ is the output over time n}{}
	\explain{$x(n)$ is the input signal}{}
	\explain{$g$ is the depth of the flanging effect}{}
	\explain{$M(n)$ is the length of the delay line at n samples}{}
\stopexplain

The delay line, $M(n)$, is typically varied according to a triangular or sinusoidal waveform and the delay length is modulated by a \gls{lfo}. The length of the delay is typically at a maximum of 15 ms as seen in \autoref{tab:delay_times}.  

The frequency response of the output is comb shaped as seen in \autoref{fig:comb_filter_response}.  

\begin{figure}[htbp]
	\centering
	\includegraphics[width=0.5\textwidth]{figures/analysis/comb_filter_math_response}
	\caption{Frequency response of a comb filter without scaling and numbers to visualize a comb filter.}
	\label{fig:comb_filter_response}
\end{figure}

For $g>0$ there are $M$ peaks in the frequency response which are centered around the frequencies calculated in \autoref{eq:comb_filter_freqs}.

\begin{equation}
\omega_{k}^{(p)}=k\cdot \frac{2\pi}{M}
\label{eq:comb_filter_freqs}
\end{equation}

\startexplain
\explain{k = 0,1,2,...,M-1}{}
\explain{M is the amout of notches}{}
\explain{p is the number of the specific peak}{}
\stopexplain

When $g =1$ the notches are at maximum attenuation with $M$ notches between the peaks at frequencies calculated in \autoref{eq:notch_freqs}.

\begin{equation}\label{eq:notch_freqs}
	\omega_k^{n}=\omega_k^{p}+\frac{\pi}{M}
\end{equation}

When $M$ varies the comb teeth squeezes in and out like the pleated layers on an accordion which produced the flanger effect as done in the 1960's with tape machines. \citep{flanger_descr}