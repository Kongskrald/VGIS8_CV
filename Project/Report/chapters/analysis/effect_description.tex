%!TEX root = ../../master.tex
\section{Effect Description}\label{sec:effect_descr}
The analysis of the effects concerns several parameters of the effects and functionality of previous designs. This also includes explanations of the effect sound and what this is supposed to sound like.

\subsection{Delay Based Effects}
Delay is the act of holding a signal to be played back after a variable amount of time. 
Different lengths of delay combined with different couplings gives different effects. Flanger, chorus, echo and reverb are all examples of delay effects. These different delay based effects are defined as having a delay range, as seen in \autoref{tab:delay_times}. This means that all the different delays in the effect are in the delay range, in order to be defined as that specific effect.

\begin{table}[htbp]
\centering
\caption{Delay length for different kinds of effects \citep{DAFX}.}
\label{tab:delay_times}
\begin{tabular}{|l|l|l|}
\hline
\rowcolor{lightgray}
delay type   & delay length       & Effect type          \\ \hline
short delay  & \textless 15 ms    & flanger \\ \hline
medium delay & 15 ms - 50 ms      & chorus \\ \hline
long delay   & \textgreater 50 ms & echo, reverberation  \\ \hline
\end{tabular}
\end{table}

\subsection{Echo}
The echo effect is a delay based effect. Echo gives a repeating but decaying signal. This is implemented using a feedback loop with a delay as shown on \autoref{fig:echo_block_anal}. The delay part of the echo can be implemented with cascades of sample hold circuits, also known as a bucket brigade device, where the signal is sampled and sent through the system. The delay of the signal can be varied by changing the clocking frequency or the length of the bucket brigade. The clock is used to toggle the switches so the signal can move through the circuit. The switches open in sets where every other is opened while the rest is closed. Increased frequency gives a smaller delay while a shorter switching frequency gives a longer delay. The minimum sample frequency is set by the maximum frequency of the signal since it is converting the signal to discrete time in order to delay it \citep{se_bucket_brigade}. In \autoref{fig:bucketbrigade} a simple bucket brigade is illustrated.

\begin{figure}[htbp]
    \centering
    \includegraphics[width=0.8\textwidth]{block_echo}
    \caption{Block diagram of the echo effect.}
    \label{fig:echo_block_anal}
\end{figure}

\ctikzset{bipoles/length=1cm}
\begin{figure}[H]
    \centering
    \includegraphics[width=\textwidth]{bucketbrigade}
    \caption{A simple bucket brigade \citep{se_bucket_brigade}}{
    \label{fig:bucketbrigade}
    }
\end{figure}


Another way of implementing a delay is with two or more magnetic tape recorders with a shared tape. In this case the audio is recorded onto a tape loop with a magnetic write head. Further down the tape there is located one or more read heads which play the audio. The delay can be adjusted by changing the distance between the read and write heads or by changing the speed of the tape \citep{mag_tape_delay_hist}.

In order to turn these delays into a echo effect the original signal need to be added to the original signal. This can be done by using a voltage adder as shown in \autoref{fig:analog_echo}.

\begin{figure}[htbp]
    \centering
    \includegraphics[width=0.6\textwidth]{analog_echo}
    \caption{Implementation of analogue echo.}
    \label{fig:analog_echo}
\end{figure}


%!TEX root = ../../master.tex
\subsection{Flanger}
\label{ssub:flanger}

The flanger effect was introduced in the 1960's and was originally an analogue effect. The effect was made by using two tape machines playing two different tapes with the same music but in sync and the output was summed together. To create the effect a flange on one tape machine was touched lightly to create a small delay between the two tapes. The flange was then released and the same procedure was done on the other tape machine to counter the delay and create a delay on the other tape. The effect is often described as the sound of a jet plane passing by which makes the listener hear the direct sound from the plane but also a reflected sound from the ground beneath. 

Nowadays the flanger effect is mostly created digitally, as described in the block diagram in \autoref{fig:flanger_blockdiagram} and is modeled as a feed forward comb filter with a delay, $M$, which is varied over time. The output from the filter will then be as in \autoref{eq:flanger_out}.

\begin{figure}[htbp]
	\centering
	\includegraphics[width=0.8\textwidth]{figures/analysis/flanger_blockdiagram.pdf}
	\caption{Flanger block diagram.}
	\label{fig:flanger_blockdiagram}
\end{figure}

\begin{equation}
	y(n)=x(n)+g \cdot x(n-M(n))
	\label{eq:flanger_out}
\end{equation}

\startexplain
	\explain{$y(n)$ is the output over time n}{}
	\explain{$x(n)$ is the input signal}{}
	\explain{$g$ is the depth of the flanging effect}{}
	\explain{$M(n)$ is the length of the delay line at n samples}{}
\stopexplain

The delay line, $M(n)$, is typically varied according to a triangular or sinusoidal waveform and the delay length is modulated by a \gls{lfo}. The length of the delay is typically at a maximum of 15 ms as seen in \autoref{tab:delay_times}.  

The frequency response of the output is comb shaped as seen in \autoref{fig:comb_filter_response}.  

\begin{figure}[htbp]
	\centering
	\includegraphics[width=0.5\textwidth]{figures/analysis/comb_filter_math_response}
	\caption{Frequency response of a comb filter without scaling and numbers to visualize a comb filter.}
	\label{fig:comb_filter_response}
\end{figure}

For $g>0$ there are $M$ peaks in the frequency response which are centered around the frequencies calculated in \autoref{eq:comb_filter_freqs}.

\begin{equation}
\omega_{k}^{(p)}=k\cdot \frac{2\pi}{M}
\label{eq:comb_filter_freqs}
\end{equation}

\startexplain
\explain{k = 0,1,2,...,M-1}{}
\explain{M is the amout of notches}{}
\explain{p is the number of the specific peak}{}
\stopexplain

When $g =1$ the notches are at maximum attenuation with $M$ notches between the peaks at frequencies calculated in \autoref{eq:notch_freqs}.

\begin{equation}\label{eq:notch_freqs}
	\omega_k^{n}=\omega_k^{p}+\frac{\pi}{M}
\end{equation}

When $M$ varies the comb teeth squeezes in and out like the pleated layers on an accordion which produced the flanger effect as done in the 1960's with tape machines. \citep{flanger_descr}
\section{Reverb}
The reverb effect is tested both in regards to \autoref{req:DL1} and the effect description in \autoref{sec:effect_descr}.
The complete test is shown in \autoref{reverb_meas_app}. The results of the test is shown in \autoref{fig:reverb_plot}.

\begin{figure}[hbpt]
	\centering
	\includegraphics[width=0.8\textwidth]{reverb_plot}
	\caption{impulse response of the reverb effect.}
	\label{fig:reverb_plot}
\end{figure}

The plot shows the reverb effect has the characteristics described in \autoref{sec:effect_descr}. This is done by making the early reflections, which are the first delays and thereafter producing the late reflections.

The tap delay is a changeable variable. According to \autoref{req:DL1} the delay must not be smaller than \SI{50}{\milli\second} which is $2205$ samples. In \autoref{fig:reverb_plot} it is shown that the first delay is no shorter than \SI{50}{\milli\second}. Therefore the reverb is successful. 
\subsection{Chorus}
The chorus effect makes one input sound like multiple outputs. If the effect for instance is used with a singer, although only one person is singing, it will sound like multiple people singing the same tones. 

Even a unison choir is not in exact unison, which means a chorus effect should modify the copies of the original input. As according to \cite{chorus_descr}, The modifications can be:

\begin{itemize}
	\item Delay
	\item Frequency shift
	\item Amplitude modulation
\end{itemize}

To make this effect the input is split into several combinations of a delay and another kind of effect tweaking some of the other properties of the signal. A block diagram showing this is illustrated in \autoref{fig:chorus_block}.

\begin{figure}[htbp]
	\centering
	\includegraphics[width=0.8\textwidth]{chorus_block}
	\caption{Block diagram showing the chorus effect.}
	\label{fig:chorus_block}
\end{figure}

The chorus is similar to the flanger effect, but with a larger delay, which is shown in \autoref{tab:delay_times}. Chorus gives the effect of more than one guitar playing as stated but also tweaks the sound a bit like the flanger. 

The effect uses feed forward in the block diagram, which means it is suitable to use a \gls{fir} filter since the output is not in a feedback loop. This means an N order \gls{fir} can be written as:

\begin{equation}
	y[n]=b_0x[n]+b_1x[n-1]+\cdots+b_Nx[n-N]
\end{equation}

\startexplain
	\explain{$x[n]$ is the input signal.}{\noSIunit}
	\explain{$y[n]$ is the output signal.}{\noSIunit}
	\explain{$N$ is the filter order.}{\noSIunit}
	\explain{$b_i$ is the amplification of the impulse response.}{\noSIunit}
\stopexplain




\subsection{Distortion and Overdrive}\label{subs:dist_overdrive}
Distortion and overdrive are two different effects but still in the same category of amplitude effects. Overdrive is dependent of the amplitude and the amount of distortion will correspond to the amplitude of the input. Whereas distortion does not depend on the amplitude, and distort the signal to a level set by the user \citep{guitar_science2014}.

When distorting a signal the waveform of the signal is changed. The distortion effect is a hard clipping of the signal and the overdrive effect is a soft clipping of the signal. This is also illustrated in \autoref{fig:clipping_waveform}.

\begin{figure}[htbp]
    \centering
    \includegraphics[width=0.8\textwidth]{clipping_waveform.pdf}
    \caption{Illustration of soft and hard clipping of a signal \citep{clipping_fig}.}
    \label{fig:clipping_waveform}
\end{figure}

A simple block diagram of a distortion effect is made. This is illustrated in \autoref{fig:dist_block}.

\begin{figure}[htbp]
    \centering
    \includegraphics[width=0.6\textwidth]{distortion_block}
    \caption{A simple block diagram of a distortion effect.}
    \label{fig:dist_block}
\end{figure}

Both distortion and overdrive have a very lose definition, which is basicly that the signal is either clipped softly or hard. Most distortion and overdrive pedals are also analogue, and usually not implemented in a digital system. Therefore measurements on an analogue distortion pedal is made, to see the output compared to the input in one version of a distortion pedal. These measurements are shown in \autoref{cha:measuring_distortion_pedal_output}. 