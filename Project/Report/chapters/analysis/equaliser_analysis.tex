%!TEX root = ../../master.tex
\subsection{Equaliser} \label{subs:equaliser_analysis}
An equaliser is a way of shaping the audio spectrum by amplifying or attenuating specific frequency bands while other remain untouched. To create an equaliser a series of first and second order shelving and peak filters as seen in \autoref{fig:equaliser_plot} are needed. The figure shows both a plot with an example of different settings for the different filters and a block diagram with the different variables for each filter. 

\begin{figure}[htbp]
	\centering
	\includegraphics[width=0.8\textwidth]{figures/analysis/EQ_plot.pdf}
	\caption{Series connection of shelving and peak filters with a shelf at the lowest and highest frequency and a band pass filter in the middle \citep{DAFX}.} 
	\label{fig:equaliser_plot}
\end{figure}

As described in the block diagram in \autoref{fig:equaliser_plot} the shelving filter has a cut-off frequency ($f_c$) and a gain ($G$) while the peak filter has a cut-off frequency ($f_c$), a bandwidth ($f_b$) and a gain ($G$). These variables define how audible the equalisation is by changing the amplification of the specified frequencies and also define how wide a range the filter is in the audio spectrum. 

An often used filter type is the constant Q filter. The Q-value is defined by the ratio between the bandwidth and the cut-off frequency and can be calculated by $Q=\frac{f_b}{f_c}$ \citep{DAFX}. This means that when the filter is adjusted by changing the cut-off frequency the ratio between the two remains the same. Another type of filter is when the Q-value is adjustable too. This means that it is possible to change the ratio between the cut-off frequency and the bandwidth which makes the slope steeper or flatter. This makes the user able to choose a more specific frequency to adjust without interfering with the nearby frequencies. 

