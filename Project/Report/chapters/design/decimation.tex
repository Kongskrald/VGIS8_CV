%!TEX root = ../../master.tex

\section{Decimation}
Decimation is turning a discrete time signal into another discrete signal which consists of a sub set of samples of the original signal. Decimation is also known as down-sampling and is defined by \autoref{eq:def_decimation}. With decimation the sample rate is reduced from ${\mathrm{F}}_{\mathrm{s}}$ to $\frac{\mathrm{F}_{\mathrm{s}}}{\mathrm{M}}$ because every $M$ sample is removed. Before the samples are removed the signal must be filtered to remove potential aliasing. This anti aliasing filter is a low-pass filter with a cut-off frequency close to the Nyquist rate of the new sample rate \citep{multirate_dsp}.

\begin{equation}\label{eq:def_decimation}
	y[n] = v[nM] =  \sum_{k=- \infty}^\infty h[k]x[nM-k]
\end{equation}

Decimation can be useful when using FIR filters since it brings low frequencies up in frequency. Making the FIR filters have less coefficients.