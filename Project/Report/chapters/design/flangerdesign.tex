%!TEX root = ../../master.tex
\section{Flanger}
\label{sec:flanger_design}
In \autoref{sec:effect_descr} the flanger effect is described. The design of the effect is made in this section. This is in regards to both the requirements and the analysis. By firstly simulating the design in MATLAB the effect is easier to tweak and afterwards implement on the \gls{dsp}.

\subsection{General Algorithm}

When designing the flanger effect, the block diagram from \autoref{ssub:flanger} is used. This diagram are shown in \autoref{fig:flanger_blockdiagram_design}.


\begin{figure}[htbp]
	\centering
	\includegraphics[width=0.8\textwidth]{figures/analysis/flanger_blockdiagram.pdf}
	\caption{Flanger block diagram.}
	\label{fig:flanger_blockdiagram_design}
\end{figure}

From this diagram, a differential equation is made, and is shown in \autoref{eq:flangerdesign_out}.

\begin{equation}
	y(n)=x(n)+g \cdot x(n-M(n))
	\label{eq:flangerdesign_out}
\end{equation}

The equation describes how the original signal is sent straight through the block through a feed forward line. The signal is sent through a delay line with a variable delay, a gain block, and finally it is summed with the original signal.

\subsection{Simulation}

The differential equation is implemented in MATLAB shown in \autoref{code:simulation_matlab_flanger}, and a simulation of the flanger effect is made. The flanger effect is made using just one comb filter and a delay. The delay of the flanger is a periodically varying delay. This is done by having a pre delay which is chosen to be approximately \SI{1}{\milli \second}. The delay is then modulated using an \gls{lfo}, which in this case is a cosine signal with a frequency of \SI{1}{\hertz} and an amplitude called depth of approximately 31 samples, which is equivalent to a modulation factor of \SI{0.7}{\milli \second}. This means that the delay will be maximally approximately \SI{1.7}{\milli \second}. The input signal in the simulation consists of white noise that is affected by the flanging effect. This is then plotted in a spectrogram.

\begin{lstlisting}[caption={Simulation of flanger in MATLAB.},language=MATLAB,label={code:simulation_matlab_flanger}]
len = 5*44100; %Length of signal
sig = white_noise(len); %Generate white noise of len samples
out = zeroes(len,1);
a = 1;
preD = round(1e-3*Fs); %pre delay 
depth = round(0.7e-3*Fs); %Depth of flanger 
for i=depth+preD+1:1:len
    delay(i)=round(preD + depth*cos(2*pi*i./((44100)))); % Current delay is calculated
    out(i)=sig(i)+a*sig(i-delay(i)); % current sample added to delayed sample
end
spectrogram(out,1024,'yaxis');
\end{lstlisting}

The spectrogram from the simulation is shown in \autoref{fig:freq_flanger}. The variable delay in the comb filter generates spikes and troughs in the frequency domain, and the variable delay changes the spacing of these on the frequency axis over time. 



\begin{figure}[htbp]
	\centering
	\includegraphics[width=0.8\textwidth]{freq_flanger_v2.eps}
	\caption{Flanger frequency response.}
	\label{fig:freq_flanger}
\end{figure}

It is shown how the output, has comb like spikes in the frequency spectrum plot. This is what yields the flanger effect to the sound. The spikes are the result of the delayed signal being in phase with the original signal, and the troughs are where the two signals are out of phase. This is all a result of the comb filter in the flanger. 
