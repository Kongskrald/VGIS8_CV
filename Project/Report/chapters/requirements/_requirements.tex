\chapter{Requirements}\label{ch:req}
To be able to measure the quality of the identification method made, a set of requirements are presented. These are produced on the background of knowledge presented in \autoref{cha:Research}.

The context of the work carried out is identity verification on smart phones. Therefore, there are certain requirements to the system. Though some smartphones nowadays are equipped with \gls{nir} cameras not all are. It is also beneficial when successful identity verification can be done using the normal \gls{vl} front camera that most smartphones are equipped with, as this requires less sensors and thus is cheaper for the smart phone manufacturers. From these considerations two requirements for the system arrises. The input images used for the identification has to be \gls{vl} images, and furthermore, the images have to have a resolution which is low enough to represent the images that would be captured with a smart phone front camera.

The designed solutions must have an accuracy comparable to the ones of state of the art solutions presented in \autoref{cha:Research}. The accuracies presented are generally at $99\%$ or above, which means the solution implemented in this project must have an accuracy equal to or above this accuracy as well. 



As the project seeks to make a better verification for security measures, the fused networks should perform better than that of the non-fused.

