\chapter{Project Specification}\glsresetall
This chapter specifies the scope of the project. It outlines and delimits the goals for the work conducted, as well as setting the requirements for the solutions implemented during the project work. 

\section{Delimitation}\label{sec:deliproject}

(In this Section: Delimit from computational power and computational time. State a central problem)   


\section{Requirements}\label{ch:req}
To be able to measure the quality of the identification method made, a set of requirements are presented. These are produced on the background of knowledge presented in \autoref{cha:Research}.

The context of the work carried out is identity verification on smart phones. Therefore, there are certain requirements to the system. Although some smartphones nowadays are equipped with \gls{nir} cameras not all are. Therefore, it would be beneficial if successful identity verification proved possible when using the normal \gls{vl} front camera, which most smartphones are equipped with, as this requires less sensors and thus is cheaper for the smart phone manufacturers. From these considerations two requirements for the system arises. The input images used for the identification has to be \gls{vl} images, and furthermore, the images have to have a resolution, which is low enough to represent the images that would be captured with a smart phone front camera.

Furthermore, the system has to identify using one of the biometric modalities face, or iris, or both, since those are the most reliable non-invasive biometric traits, which can be obtained by a smart phone and especially with a \gls{vl} camera. 

The designed solutions must have an accuracy comparable to the ones of state of the art solutions and commercial systems presented in \autoref{cha:Research}. The accuracies presented are generally at $99\%$ or above, which means the solution implemented in this project must have an accuracy equal to or above this accuracy as well. 

The solution for identification based on fusion of the two modalities has to have an accuracy higher than the accuracies of the system based on the individual modalities for it to be accepted. If the accuracy is the same, there is no gain in the increased computational complexity of a multimodal system and as a result the system will be forsaken.\todo[inline]{Add delimitation from computational limitations and requirements for the computational time} 
%As the project seeks to make a better verification for security measures, the fused networks should perform better than that of the non-fused.

