\chapter{Methods}
This chapter describes the aim of the mini project as well as the methods used.

\section{Approach} 
To implement the functionality from \autoref{sec:The_Functions}, Python will be be used. The system was be designed according to the following plan:

\begin{enumerate}
	\item The system specifications will be specified
	\item The code will be compartmentalised into modules
	\item During implementation unit testing will be used
	\item Once it passes the unit testing it will be optimised
\end{enumerate}

The code will be differentiated into two parts; the na\"ive and the optimised. The na\"ive implementation will be written using only functionality available in Python. The optimised version will contain the same functionality as the na\"ive but will be implemented using optimised libraries like numpy.

To track the progress git will be used for version control and feature/bug tracking. Comments in the code will be used as documentation. Once the system works and has been optimised, it will be rewritten for parallel processing to see if it can be optimised further. 


\section{Test setup}
All tests is conducted on the same laptop in order to be as consistent as possible with the conditions under which the testing is made. The laptop used is a MacBook Pro from early 2013, running High Sierra. The MacBook has 8 GB ram and has a 2.6 GHz Intel Core i5-3230M processor, which is a dual core processor. The scripts are run using the Python version 3.6.3. 